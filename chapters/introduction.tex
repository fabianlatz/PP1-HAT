\chapter{Einleitung}
In der modernen Softwareentwicklung spielt die effiziente Zusammenarbeit zwischen verschiedenen Teams und Komponenten eine entscheidende Rolle. Dabei gewinnt der \acs{API}-First-Ansatz zunehmend an Bedeutung, insbesondere im Kontext von \ac{REST}-\acp{API}. Diese Methodik verspricht, die Entwicklung von Schnittstellen zu optimieren und die Kooperation zwischen Entwicklern zu verbessern.

Die Komplexität verteilter Systeme stellt Entwickler vor große Herausforderungen. Vor allem die Gestaltung von \acp{API}, die als Bindeglied zwischen verschiedenen Softwarekomponenten fungieren, erfordert besondere Aufmerksamkeit. Häufig wird die \ac{API}-Entwicklung jedoch vernachlässigt oder als nachrangig betrachtet, was zu suboptimalen Schnittstellen führen kann~\autocite[73]{bea22}.

Agile Entwicklungsteams stoßen bei der Anwendung einer serviceorientierten Architektur auf Schwierigkeiten. Es besteht eine Diskrepanz zwischen kundenorientierten Anforderungen und der tatsächlichen Anwendungsarchitektur, was zu schlecht gestalteten \acp{API} führen kann. Die Anwendung von Best Practices für eine saubere, benutzerfreundliche \ac{API} hängt stark von der Erfahrung des Entwicklungsteams ab~\autocite[10]{riv13}.

Ein gutes \ac{API}-Design ist ein kritischer Aspekt moderner Softwareentwicklung und eine grundlegende Voraussetzung für den \ac{API}-First-Ansatz. Dennoch befindet sich der \ac{API}-First-Ansatz als Methodik noch in den Anfängen~\autocite[75,78]{bea22}.

Die zentrale Forschungsfrage dieser Arbeit lautet:
\textquote{Wie kann der \ac{API}-First-Ansatz eine effiziente Zusammenarbeit in der Softwareentwicklung auf Basis einer \acs{REST}-\acs{API}-Spezifikation fördern?}

Um diese Frage umfassend zu beantworten, werden folgende Aspekte untersucht:
\begin{enumerate}
	\item Welche Vorteile bietet der \acs{API}-First-Ansatz gegenüber traditionellen Entwicklungsmethoden?
	\item Wie kann OpenAPI als Spezifikationsstandard im Rahmen des \acs{API}-First-Ansatzes eingesetzt werden?
	\item Welche Best Practices existieren für eine erfolgreiche Implementierung des \acs{API}-First-Ansatzes?
\end{enumerate}

Das Ziel dieser Arbeit ist es, die Bedeutung und Anwendung des \acs{API}-First-Ansatzes in der modernen Softwareentwicklung zu analysieren und dessen Potenzial zur Förderung effizienter Zusammenarbeit aufzuzeigen.

Die vorliegende Arbeit basiert auf einer umfassenden Literaturrecherche und -analyse. Dabei werden sowohl wissenschaftliche Publikationen als auch praxisorientierte Quellen berücksichtigt, um einen ganzheitlichen Überblick über den aktuellen Stand der Forschung und Anwendung des \acs{API}-First-Ansatzes zu gewinnen.

Die Untersuchung gliedert sich in mehrere Teile: Zunächst werden die Grundlagen des \acs{API}-First-Ansatzes erläutert, gefolgt von einer Betrachtung von OpenAPI als Spezifikationsstandard. Anschließend wird analysiert, wie der \acs{API}-First-Ansatz die Zusammenarbeit in der Softwareentwicklung verbessern kann. Abschließend werden Best Practices für eine \acs{API}-First-Entwicklung vorgestellt.

Durch die Verknüpfung theoretischer Konzepte mit praktischen Anwendungsbeispielen soll ein umfassendes Verständnis des \acs{API}-First-Ansatzes und seiner Bedeutung für die moderne Softwareentwicklung vermittelt werden.
