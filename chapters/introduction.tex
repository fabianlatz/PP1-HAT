\chapter{Einleitung}
\label{ch:introduction}
In der modernen Softwareentwicklung spielen \acp{API} eine zentrale Rolle, insbesondere in Service"=orientierten Architekturen und Microservices"=Umgebungen.
Trotz ihrer stetig wachsenden Bedeutung sind unvollständige oder unklare \ac{API}"=Dokumentationen häufig eine Quelle für Missverständnisse und Fehler in Softwareprojekten.
Dieses Problem wird durch traditionelle Entwicklungsmethoden, wie den Code"=First"=Ansatz, verschärft, bei dem die \ac{API}"=Spezifikation erst nach der Implementierung erstellt wird~\cite[73]{bea22}.

Die vorliegende Arbeit befasst sich mit dem \AFA, der die Spezifikation der \ac{API} als zentrales Artefakt der Softwareentwicklung an den Beginn des Entwicklungsprozesses stellt.
Die Untersuchungen stehen konkret unter der Leitfrage: \emph{Wie kann der \AFA eine effiziente Zusammenarbeit in der Softwareentwicklung auf Basis einer \acs{REST}"=\acs{API}"=Spezifikation fördern?}

Der Ansatz steckt, trotz der Allgegenwärtigkeit von \acp{API} in der Softwareentwicklung und der Notwendigkeit für effiziente Methoden zur \ac{API}"=Entwicklung, noch in den Kinderschuhen~\cite[78]{bea22}.
Ziel dieser Arbeit ist es daher, die Vorteile des \AFAes zu untersuchen und Best Practices für dessen Umsetzung herauszuarbeiten.
Durch die Analyse der Rolle von \OA im \AFA soll ein umfassendes Verständnis für die bereits existierenden Möglichkeiten dieses Ansatzes geschaffen werden.
Letztlich soll die Arbeit einen Beitrag zur Verbesserung der Softwareentwicklung durch effiziente Zusammenarbeit und höhere Code"=Qualität leisten.

\paragraph{}
Bevor die Grundlagen des \AFAes gelegt werden können, folgt zunächst eine Definition des Begriffs \acf{API} sowie eine Einschränkung auf die in dieser Arbeit betrachtete Form von \acp{API}:

Eine \ac{API} ist eine Schnittstelle, die es Anwendungen oder Komponenten einer Anwendung ermöglicht, miteinander zu kommunizieren.
Dazu greifen \acp{API} auf eine Reihe von Regeln, Protokollen und Werkzeugen zurück, um diese Schnittstellen zu spezifizieren.
Sie abstrahieren dabei die Komplexität der beteiligten Systeme durch die Bereitstellung einer vereinfachten Variante der Funktionalität, die von anderen Systemen genutzt werden kann~\cite[1]{kul23}.
Eine \ac{API} ähnelt einer Benutzeroberfläche, die die Interaktionsmöglichkeiten eines Benutzers mit dem System bereitstellt, jedoch auf die Interaktion zwischen Softwarekomponenten beschränkt.
Statt Textfeldern, Knöpfen und Reglern stellt eine \ac{API} Endpunkte, mögliche Parameter und zu erwartende Rückgabewerte bereit, um Daten auszutauschen~\cites[351]{de23}{ope23a}.

\acp{API} werden häufig mit Verträgen verglichen, da sie ähnlich strikte Vorgaben stellen und eine ähnliche Funktion erfüllen:
Der Anbieter einer \ac{API} legt die Regeln fest, unter deren Einhaltung er die \ac{API} kontinuierlich bereitstellt und nicht ohne Vorankündigung ändert oder abschaltet.
Der Nutzer kann sich bei seiner Implementierung darauf verlassen und verpflichtet sich im Gegenzug, die \ac{API} nur wie vorgegeben einzusetzen~\cites[1627]{cha21}{ope23a}.
% Rückbezug bei Parallelisierbarkeit

In dieser Arbeit werden lediglich \acs{REST}"=\acp{API} betrachtet, da sie auf dem \acf{HTTP} als gemeinsame Sprache der Kommunikation basieren und damit in verschiedenen Anwendungsfällen verbreitet sind.
Die Grundlagen für ein gutes \ac{API}"=Design berufen sich damit ebenso auf die Standards von \ac{HTTP} und Prinzipien von \ac{REST}~\cite[1628]{cha21}.