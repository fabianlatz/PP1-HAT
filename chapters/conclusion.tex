\chapter{Fazit und zukünftige Forschung}
In dieser Arbeit wurde der \AFA als zentrales Konzept zur Förderung effizienter Zusammenarbeit in der Softwareentwicklung untersucht.
Der Fokus lag dabei auf der Rolle von \OA als Spezifikationsstandard für \ac{REST}"=\acp{API}.
Wie in der \hyperref[ch:introduction]{Einleitung} dargelegt, sind unvollständige oder unklare \ac{API}"=Dokumentationen häufig eine Quelle für Missverständnisse und Fehler in Softwareprojekten.
Der \AFA adressiert dieses Problem, indem er die \ac{API}"=Spezifikation als eigenständiges Artefakt zu Beginn des Entwicklungsprozesses priorisiert.

Durch die frühzeitige Spezifikation der \ac{API} können Entwickler Design"=Fehler erkennen und beheben, was zu einer verbesserten Code"=Qualität führt.
Zudem wird dadurch eine parallele Entwicklung von Client"= und Server"=Komponenten ermöglicht, was die Kommunikation zwischen Entwicklern verbessert und die Softwareentwicklung beschleunigt.

\OA bietet hierfür eine Programmiersprachen-unabhängige Grundlage, die die Erstellung, Validierung und Verwendung von \ac{API}"=Spezifikationen erleichtert.
Durch die Integration von \OA in den \AFA können Entwickler auf eine standardisierte und maschinell lesbare Beschreibung der \ac{API} zurückgreifen, was die Zusammenarbeit und die Wiederverwendbarkeit von Code fördert.

Die zentrale Bedeutung von \acp{API} in der modernen Softwareentwicklung, insbesondere in Service"=orientierten Architekturen und Microservices"=Umgebungen, unterstreicht die Notwendigkeit eines strukturierten Ansatzes zur \ac{API}"=Entwicklung.
Der \AFA fördert die Anforderungen an diese Architekturen, vor allem Wartbarkeit und Skalierbarkeit, gezielt und beschleunigt die Entwicklung bis zur Marktreife, was ihn auch wirtschaftlich interessant machen dürfte.

Zukünftige Forschung könnte sich auf die weiteren Potenziale des \AFAes in Kombination mit anderen modernen Entwicklungsparadigmen konzentrieren.
Die Untersuchung der wirtschaftlichen Auswirkungen einer umfassenden Einführung des \AFAes in Unternehmen könnte ebenfalls wertvolle Erkenntnisse liefern.
Dabei spielen besonders die verkürzte Entwicklungsdauer durch Parallelisierbarkeit und Automatisierung verschiedener Arbeitsschritte eine zentrale Rolle, um die Kundenzufriedenheit und Effizienz von Softwareprojekten zu verbessern~\cites[77]{bea22}[355,360]{de23}.
Für diese Effekte ist ein Umdenken in der Branche notwendig, um Softwarearchitekturen und Entwicklungsprozesse an \AF auszurichten~\cites[2]{kul23}[361]{de23}.
Insgesamt trägt der \AFA maßgeblich zur Verbesserung der Softwareentwicklung bei, indem er die Effizienz und Qualität der Zusammenarbeit zwischen Entwicklern steigert.