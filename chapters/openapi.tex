\chapter{\OA als Spezifikationsstandard}
In Softwareprojekten sind unvollständige, unklare oder schlicht fehlende Dokumentationen von \acp{API} eine häufige Ursache für Missverständnisse und Fehler~\cite{ope23a}.
Um diese Probleme zu vermeiden, gibt es verschiedene Standards und Konventionen, die die Dokumentation von \acp{API} vereinheitlichen und automatisieren.

Der aktuell am besten etablierte Standard für \ac{HTTP}-basierte \acp{API} ist \OA.
Die Struktur und Syntax einer \ac{API} wird dafür in einer Programmiersprachen- und Anbieter-unabhängigen \ac{OAD} im \acs{YAML}- oder \acs{JSON}-Format festgehalten, die konform zur \ac{OAS} sein muss.
Damit ist eine \ac{OAD} ist von Menschen als auch maschinell lesbar und lässt sich dementsprechend  einfach verarbeiten und unter Versionskontrolle stellen.

Die \ac{OAS}, die das Format einer validen \ac{OAD} beschreibt, wird quelloffen von der \ac{OAI} entwickelt, die ihrerseits ein Projekt der Linux Foundation ist.
Sie beinhaltet einige Begriffe, um allgemein-verständliche und weit-verbreitete Konzepte rund um \acp{API} im Kontext von \acs{HTTP} und \acs{JSON} zu beschreiben.
Darüber hinaus hat sich um \OA eine Reihe von Werkzeugen und Bibliotheken entwickelt, die die Erstellung, Validierung und Verwendung von \acp{OAD} erleichtern~\cites{ope,ope23,ope23a}.

\section{Möglichkeiten mit \OA}

\section{Rolle von \OA im \AFA}
