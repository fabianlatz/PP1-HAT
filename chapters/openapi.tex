\chapter{\OA als Spezifikationsstandard} \label{ch:openapi}
In Softwareprojekten sind unvollständige, unklare oder schlicht fehlende Dokumentationen von \acp{API} eine häufige Ursache für Missverständnisse und Fehler~\cite{ope23a}.
Um diese Probleme zu vermeiden, gibt es verschiedene Standards und Konventionen, die die Dokumentation von \acp{API} vereinheitlichen und automatisieren.

Der aktuell am besten etablierte Standard für \ac{HTTP}"=basierte \acp{API} ist \OA\@.
Die Struktur und Syntax einer \ac{API} wird dafür in einer Programmiersprachen"= und Anbieter"=unabhängigen \ac{OAD} im \acs{JSON}"= oder \acs{YAML}"=Format festgehalten, die konform zur \ac{OAS} sein muss.
Damit ist eine \ac{OAD} ist von Menschen als auch maschinell lesbar und lässt sich dementsprechend  einfach verarbeiten und unter Versionskontrolle stellen.

Die \ac{OAS}, die das Format einer validen \ac{OAD} beschreibt, wird quelloffen von der \ac{OAI} entwickelt, die ihrerseits ein Projekt der Linux Foundation ist.
Sie beinhaltet einige Begriffe, um allgemein"=verständliche und weit"=verbreitete Konzepte rund um \acp{API} im Kontext von \acs{HTTP} und \acs{JSON} zu beschreiben.
Darüber hinaus hat sich um \OA eine Reihe von Werkzeugen und Bibliotheken entwickelt, die die Erstellung, Validierung und Verwendung von \acp{OAD} erleichtern~\cites{ope,ope23,ope23a}.

\section{Möglichkeiten mit \OA}
Neben der bereits angesprochenen Generierung einer \ac{OAD} bietet das Ökosystem rund um \OA eine Reihe von Funktionen, die bei der Spezifikation und Implementierung einer \ac{API} helfen können:
\begin{enumerate}
	\item \textbf{Validierung und statische Analyse:} Die \ac{OAD} kann auf syntaktische Korrektheit und Einhaltung der \ac{OAS} geprüft werden.
	\item \textbf{Datenvalidierung:} Die \ac{OAD} kann auch auf semantische Korrektheit geprüft werden, indem sie mit den tatsächlichen Daten verglichen wird, die die \ac{API} zurückgibt.
	\item \textbf{Grafische Editoren:} Eine Reihe von Werkzeugen bietet grafische Oberflächen, um \acp{OAD} zu erstellen und zu bearbeiten.
	\item \textbf{Code"=Generierung:} Mithilfe von Bibliotheken für viele Server"= sowie Client"=seitige Programmiersprachen und Frameworks kann aus der \ac{OAD} ein Code"=Grundgerüst generiert werden, das automatisch konform zur gegebenen \ac{OAD} ist.
	\item \textbf{\hyperref[sec:mock-server]{Mock"=Server:}} Neben der Generierung des Codes zur weiteren Entwicklung können auch Mock"=Server generiert werden, die die \ac{API} simulieren und so die Entwicklung von Clients erleichtern. Damit wird auch eine testgetriebene Entwicklung unterstützt.
	\item \textbf{Sicherheitsanalyse:} Bereits in der Design"=Phase können Sicherheitsrisiken identifiziert und bestenfalls behoben werden.
\end{enumerate}\cite{ope23}

\OA bietet damit einen leicht verständlichen, Programmiersprachen"= und Anbieter"=unabhängigen Einstieg in jede \ac{API}, die nach ihr spezifiziert wurde.
So erklärt sich auch die hohe Verbreitung bis hin zum Status als Industriestandard für \acs{REST}"=\acs{API}"=Spezifikationen~\cite{ope}.

\section{Rolle von OpenAPI im API-First-Ansatz}
Die \ac{OAI} selbst legt für den Einsatz von \OA keine spezifischen Anwendungsfälle fest, empfiehlt allerdings einen \AFA, in dem die \ac{OAD} als erstes erstellt wird, um darauf die Implementierung der \ac{API} aufzubauen~\cites{ope,ope24}.
Bei einem solchen Vorgehen ist die Unabhängigkeit von Programmiersprachen und Frameworks besonders hilfreich, da so eine nutzerzentrierte Beschreibung der \ac{API} von Beginn an ermöglicht wird, statt auf Besonderheiten einzelner Implementierungen Rücksicht nehmen zu müssen~\cite{ope}.
Dabei erleichtern Werkzeuge aus dem \OA-Umfeld die Spezifikation ohne eigentliche Programmierung oder Kenntnis davon enorm~\cites[1629]{cha21}{ope24}.

Aufgrund ihrer zentralen Rolle für die weitere Entwicklung sollten \acp{OAD} in jedem Fall unter Versionskontrolle gestellt und damit allen Beteiligten immer aktuell zur Verfügung gestellt werden.
So können sie permanent zur Generierung von Code, Testfällen oder zur Validierung der Implementierung genutzt werden.
Für letzteres können spezielle Annotationen im Quellcode genutzt werden, um eine \ac{OAD} aus der Implementierung zu generieren und mit der vorgegebenen zu vergleichen \textendash\ beispielsweise mittels automatisierter Tests im Rahmen der Continuous Integration~\cite{ope24}.