\begin{otherlanguage}{ngerman}
	\pdfbookmark[0]{Zusammenfassung}{Zusammenfassung}
	\chapter*{Zusammenfassung}
	In der modernen Softwareentwicklung spielen APIs eine zentrale Rolle, insbesondere in Service"=orientierten Architekturen und Microservices"=Umgebungen.
	Unvollständige oder unklare API"=Dokumentationen sind jedoch häufig eine Quelle für Missverständnisse und Fehler.
	Der \AFA bietet eine Lösung, indem er die Spezifikation der API als zentrales Artefakt der Softwareentwicklung in den Vordergrund stellt.

	Diese Arbeit untersucht den \AFA und seine Rolle bei der Förderung effizienter Zusammenarbeit in der Softwareentwicklung.
	Der Einsatz von OpenAPI als Standard für REST"=API"=Spezifikationen ermöglicht eine programmiersprachenunabhängige Beschreibung der API und bietet Werkzeuge zur Validierung, Code"=Generierung und Sicherheitsanalyse.
	Die Ergebnisse zeigen, dass der \AFA die Kommunikation zwischen Entwicklern verbessert, die Code"=Qualität erhöht und die Softwareentwicklung beschleunigt.

	Durch die Analyse von Best Practices für die API"=First"=Entwicklung wird ein umfassendes Verständnis für die Umsetzung dieses Ansatzes geschaffen.
	Die Arbeit leistet damit einen Beitrag zur Verbesserung der Softwareentwicklung durch effizientere Zusammenarbeit und höhere Code"=Qualität.
	Sie bietet wertvolle Erkenntnisse für Entwickler und Unternehmen, die ihre Softwareentwicklungsprozesse optimieren möchten.
\end{otherlanguage}
